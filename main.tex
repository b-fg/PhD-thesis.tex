%% ---------------------------------------------------------------
%% author: Bernat Font Garcia
%% tex: A document containing my PhD thesis
%% ---------------------------------------------------------------
\documentclass[11pt, a4paper]{etc/feereport}
\usepackage{subfiles}
\graphicspath{{img/}{../img/}{../../img/}}
%%----------------------------------------------------------------
%% pdf authorship settings
\usepackage{hyperref}
\hypersetup{
	unicode=true, 
	pdfnewwindow=true, 
	colorlinks=true, 	% (false,true)
	pdfborder={0 0 0},	
	linkcolor=blue,
	linktoc=all, 		% (none,all) 
	citecolor=blue,
	urlcolor=blue,
	breaklinks=true
}
% \usepackage{setspace}
% \singlespacing
% \onehalfspacing
% \doublespacing
%% ----------------------------------------------------------------
%% Definitions.tex
%% ---------------------------------------------------------------- 
%% People
\newcounter{address}
\setcounter{address}{1}
\renewcommand{\theaddress}{\textsuperscript{\fnsymbol{address}}}
\newcommand{\address}[1]{\refstepcounter{address}\theaddress#1\\}
\newcommand{\Name}[3]{\texorpdfstring{\href{mailto:#3}{#2}#1}{#2}\xspace}

%% Dingbats
\newcommand{\tick}{\ding{51}}
\newcommand{\cross}{\ding{55}}

%% Calculus
\renewcommand{\pd}[2]{\frac{\partial #1}{\partial #2}}
\newcommand{\pDn}[3]{\frac{\partial^{#3}#1}{\partial#2^{#3}}\xspace}
\newcommand{\dd}[2]{\ensuremath{\frac{\mathrm{d} #1}{\mathrm{d} #2}}\xspace}
\newcommand{\ddn}[3]{\frac{\partial^{#3}#1}{\partial#2^{#3}}\xspace}
\newcommand{\pdx}[2]{\dfrac{\partial #1}{\partial #2}}
\newcommand{\pdw}[2]{\tfrac{\partial #1}{\partial #2}}
\newcommand{\avg}[1]{\left\langle #1 \right\rangle}
\newcommand{\pars}[1]{\left(#1 \right)}
\newcommand{\bracs}[1]{\left[#1 \right]}
\newcommand{\p}{%
    \mathchoice%
        {\turnbox{12}{$\displaystyle\,'$}\;}%
        {\turnbox{12}{$\textstyle\,'$}\;}%
        {\turnbox{12}{$\scriptstyle\,'$}\;}%
        {\turnbox{12}{$\scriptscriptstyle\,'$}\;}%
}%
\newcommand{\ubar}[1]{\text{\b{$#1$}}}
\newcommand{\pdn}[3]{\ensuremath{\partial^{#3} {#1}}{\partial #2^{#3}}}
\newcommand{\rb}[1]{\mathrm{\mathbf{#1}}}
\newcommand{\vect}[1]{\boldsymbol{#1}}
\newcommand{\matr}[1]{\underline{\bm{\mathit{#1}}}}
\newcommand*{\deriv}[3][]{\frac{\diff^{#1}#2}{\diff#3^{#1}}}

%% Link colors
\newcommand{\mylink}[3][blue]{\href{#2}{\color{#1}{#3}}}

%Other stuff
\newcommand{\smallgtrsim}{\smallsym{\mathrel}{\gtrsim}}
\makeatletter
\newcommand{\smallsym}[2]{#1{\mathpalette\make@small@sym{#2}}}
\newcommand{\make@small@sym}[2]{%
  \vcenter{\hbox{$\m@th\downgrade@style#1#2$}}%
}
\newcommand{\downgrade@style}[1]{%
  \ifx#1\displaystyle\scriptstyle\else
    \ifx#1\textstyle\scriptstyle\else
      \scriptscriptstyle
  \fi\fi
}
\makeatother

\newcommand\Tstrut{\rule{0pt}{2.6ex}}         % 'top' strut
\newcommand\Bstrut{\rule[-0.9ex]{0pt}{0pt}}   % 'bottom' strut

\newcolumntype{A}{ >{$} r <{$} @{} >{${}} l <{$} } % Align table cells for equations
\newcolumntype{L}[1]{>{\raggedright\let\newline\\\arraybackslash\hspace{0pt}}m{#1}}
\newcolumntype{C}[1]{>{\tiny\centering\let\newline\\\arraybackslash\hspace{0pt}}m{#1}}
\newcolumntype{R}[1]{>{\raggedleft\let\newline\\\arraybackslash\hspace{0pt}}m{#1}}
\newcolumntype{N}{@{}m{0pt}@{}}

\newcommand*{\ovB}[1]{%
  \m@th\overline{\mbox{#1\rule{0pt}{3mm}}}
}
\newcommand*{\ovA}[1]{%
  \m@th\overline{\mbox{#1}\raisebox{3mm}{}}
}
\newcommand{\specialcell}[2][c]{%
  \begin{tabular}[#1]{@{}c@{}}#2\end{tabular}}
  
\newcommand{\done}{\cellcolor{SkyBlue}}  %{0.9}
\newcommand{\hcyan}[1]{{\color{teal} #1}}

\setlist[itemize]{itemsep=3pt, topsep=0pt}
\makeatletter
\def\ignorecitefornumbering#1{%
     \begingroup
         \@fileswfalse
         #1%                     % do \cite comand
    \endgroup
}
\makeatother

\newcommand{\BibTeX}{{\rm B\kern-.05em{\sc i\kern-.025em b}\kern-.08em T\kern-.1667em\lower.7ex\hbox{E}\kern-.125emX}}

%% ---------------------------------------------------------------
\begin{document}
\frontmatter

\title{Modelling of Flow Past Long Cylindrical Structures}
\date{September 2020}
\author{Bernat Font Garcia}
\email{b.fontgarcia@soton.ac.uk}
\supervisor{Owen R. Tutty, Gabriel D. Weymouth, Vinh-Tan Nguyen}
\department{Aerodynamics and Flight Mechanics Group}
\group{Aerodynamics and Flight Mechanics Group}
\faculty{Faculty of Engineering and Physical Sciences}
\university{University of Southampton}

\maketitle

\begin{abstract}
Turbulent flows are fundamental in engineering and the environment, but their chaotic and three-dimensional (3-D) nature makes them computationally expensive to simulate.
In this work, a dimensionality reduction technique is investigated to exploit flows presenting an homogeneous direction, such as wake flows of extruded two-dimensional (2-D) geometries.
First, we examine the effect of the homogeneous direction span on the wake turbulence dynamics of incompressible flow past a circular cylinder at $Re=10^4$.
It is found that the presence of a solid wall induces 3-D structures even in highly constricted domains.
The 3-D structures are rapidly two-dimensionalised by the large-scale K\'{a}rm\'{a}n vortices if the cylinder span is 50\% of the diameter or less, as a result of the span being shorter than the natural wake Mode B instability wavelength.
It is also observed that 2-D and 3-D turbulence dynamics can coexist at certain points in the wake depending on the domain geometric anisotropy.
With this physical understanding, a 2-D data-driven model that incorporates 3-D effects, as found in the 3-D wake flow, is presented.
The 2-D model is derived from a novel flow decomposition based on a local spanwise average of the flow, yielding the spanwise-averaged Navier--Stokes (SANS) equations.
The 3-D effects included in the SANS equations are in the form of spanwise-stress residual (SSR) terms.
The inclusion of the SSR terms in 2-D systems modifies the flow dynamics from standard 2-D Navier--Stokes to spanwise-averaged dynamics.
A machine-learning (ML) model is employed to provide closure to the SANS equations.
In the a-priori framework, the ML model yields accurate predictions of the SSR terms, in contrast to a standard eddy-viscosity model which completely fails to capture the closure term structures.
The trained ML model is also assessed for different Reynolds regimes and body shapes to the training case where, despite some discrepancies in the shear-layer region, high correlation values are still observed.
In the a-posteriori analysis, while we find evidence of known stability issues with long-time ML predictions for dynamical systems, the closed SANS equations are still capable of predicting wake metrics and induced forces with errors from 1-10\%.
This results in approximately an order of magnitude improvement over standard 2-D simulations while reducing the computational cost of 3-D simulations by 99.5\%.
\end{abstract}

\begingroup 
\tableofcontents
\listoffigures
\listoftables
\endgroup

\declaration
\dedicatory{Dedicat a la meva família, la de naixement i l'escollida.}
\acknowledgements{
I am forever indebted with all my three supervisors: Gabriel D. Weymouth, Owen R. Tutty and Vinh-Tan Nguyen. 
I have learnt so much from all of you, and I could have never imagined working in a better team, both from a personal and a professional standpoint.
You have always been there for the good and (more importantly) for the bad. 
Thank you very much.

I also thank the University of Southampton and the Singapore Agency for Science, Technology and Research (A*STAR).
The collaboration between the A*STAR Institute of High Performance Computing and the Faculty of Engineering and Physical Sciences of the University of Southampton has made this work possible.

Last but not least, I want to thank all the good friends I have made in Southampton and Singapore.
Together we have supported each other and shared unforgettable moments.
Thank you for being there.
}
\nomenclature

%% ---------------------------------------------------------------
\mainmatter

\chapter{Introduction}
\subfile{chapters/chapter1}
\chapter{Theoretical background}
\subfile{chapters/chapter2}
\chapter{Span effect on the turbulence nature of flow past a circular cylinder} 
\subfile{chapters/chapter3}
\chapter{The spanwise-averaged Navier--Stokes equations}
\subfile{chapters/chapter4}
\chapter{A machine-learning model for SANS}
\subfile{chapters/chapter5}
\chapter{Conclusions and future work}
\subfile{chapters/chapter6}

\appendix

\chapter{Reynolds regimes of flow past a circular cylinder}
\subfile{chapters/appendices/appendixA}
\chapter{Validation and verification of the in-house solver}
\subfile{chapters/appendices/appendixB}
\chapter{Additional SANS derivations}
\subfile{chapters/appendices/appendixC}
\chapter{Machine-learning model hyper-parameters tuning}
\subfile{chapters/appendices/appendixD}
\chapter{Hardware and software details}
\subfile{chapters/appendices/appendixE}

\backmatter

\bibliographystyle{etc/apalike-refs}
\bibliography{references}
%% ---------------------------------------------------------------
\end{document}
