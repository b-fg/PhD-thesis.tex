%% ----------------------------------------------------------------
%% Appendix: Reynolds egimes of flow past a circular cylinder
%% ----------------------------------------------------------------
%!TeX root = subfile
\label{chapter:appendixA}
\markboth{Appendix A Reynolds regimes of flow past a circular cylinder}{Appendix A Reynolds regimes of flow past a circular cylinder}
\documentclass[../main.tex]{subfiles}
% ---------------------------------------------------------------- 
\begin{document}
The classification of flow past a circular cylinder according to its Reynolds regime is summarised in \tref{tab:Re_regimes}.
The flow remains laminar, steady and 2-D along the span up to $Re<47$ and, from this point onwards, vortex shedding starts to develop.
A supercritical Hopf bifurcation is responsible for the onset of the laminar vortex shedding \citep{Williamson1996a}.
An oblique shedding with respect to the axis of the cylinder might also be observed when perturbations are introduced into the end-boundaries of the cylinder (often arising in experiments rather than numerical simulations).
Otherwise, the wake remains 2-D up to $Re=160-180$, where the first 3-D instability at the wake (Mode A) comes into the picture.
This mode arises naturally and is characterised by streamwise vortices with a spanwise wavelength ($\lambda_z$) of approximately 4 diameters.
The shedding frequency ($f_s$) experiences a rapid decrease at the onset of this instability.
There is no general agreement on the physical mechanism driving this instability even though some authors suggest a core instability, braid instability, elliptical vortex instability or centrifugal instability \citep{Noack1999}.

\begin{table}[!htpb]
\centering
\captionsetup{width=0.82\textwidth}
  \caption{Reynolds regimes of incompressible viscous flow past a circular cylinder. Note that some Reynolds numbers might differ among authors.}
  \begin{tabular}{lAAA}
    \toprule
    \multicolumn{1}{l}{Flow phenomenon}&\multicolumn{2}{c}{Reynolds regime}&\\ 
    \midrule
    Laminar 2-D steady & &\phantom{<} \hspace{0.2cm} Re<47\\
    Laminar 2-D vortex shedding & 47 &<Re<80\\
    Mode A 3-D wake transition ($\lambda_z/D\approx4$) & 180&<Re<230\\
    Mode B 3-D wake transition ($\lambda_z/D\approx1$) & 200&<Re<10^4-10^5\\
    Mode C 3-D wake transition ($\lambda_z/D\approx2$) & 170&<Re<270\\ 
    Turbulent wake transition & 260&<Re<300\\   
    Subcritical regime & 250&<Re<2\cdot 10^5\\
    Drag crisis & 2\cdot 10^5&<Re<4\cdot 10^5\\
    Supercritical regime & 4\cdot 10^5&<Re\\        
    \bottomrule
  \end{tabular}  
  \label{tab:Re_regimes}
\end{table}

Around $Re=200-230$, Mode B takes place and Mode A gradually decreases.
The Mode B instability presents a spanwise wavelength of roughly $1D$.
Rib-like streamwise vortices are generated and the mode becomes more dominant as $Re$ increases, eventually replacing all other transition modes at $Re=260$.
During this process, the wake spanwise correlation length increases as well.
With this, it can be noted that the natural wake transition scenario is successively: 2-D vortex shedding, Mode A, Mode B.
The Mode C instability is not a natural mode since it requires direct excitation from added perturbations or specific boundary conditions.
It is characterised by a $2T_s$ vortex-shedding period, a spanwise wavelength in the order of 2 diameters, and it has been observed typically within $Re<300$.
For $Re>270$, the Mode C instability periodicity becomes less apparent as the wake transitions to a turbulent state \citep{Jiang2020}.
Further information including diagrams and flow visualisations of 3-D instability mechanisms can be found in \cite{Williamson1996b, Williamson1996a}.

It is difficult to accurately define the actual turbulence transition point on the wake and this is normally indicated through temporal averages.
The transition to turbulence is triggered by the interaction of the different wake instabilities.
With increasing Reynolds number, the subcritical regime takes place.
A sharp decrease of the lift force on the cylinder (increase on base suction) is found as well as a decrease in the Strouhal number.
Up to $Re=5\cdot 10^3$, the transition appears to be caused by rib-like vortices of Mode B-type and the transition point is in a fixed streamwise position with respect to the cylinder.
As the Reynolds number increases, the transition is driven by 3-D structures generated through the interaction of the K\'{a}rm\'{a}n vortices with the separated shear layer, a.k.a. shear-layer transition regime.
This results into an upstream displacement of the transition point \citep{Bloor1964,Williamson1996a,Norberg2001}.

The drag crisis is observed at $Re=2\cdot 10^5$ and arises because of the cylinder boundary layer becoming turbulent.
This causes the separated shear layer to narrow as the separation points are moved towards the rear part of the cylinder.
A sharp reduction in drag and base suction can be appreciated.
Laminar separation bubbles can also be observed on the cylinder surface.
Finally, above $Re=4\cdot 10^5$, the supercritical (and transcritical for $Re>10^6$) regime is defined.
The wake is further narrowed and the shedding frequency increases.
Asymmetrical (laminar/transitional) states of separation bubbles at both upper and lower sides of the cylinder can be observed \citep{Noack1999b}.
% ---------------------------------------------------------------- 
\end{document}