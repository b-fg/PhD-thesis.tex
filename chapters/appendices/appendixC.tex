%% ----------------------------------------------------------------
%% Appendix: Additional SANS derivations
%% ----------------------------------------------------------------
%!TeX root = subfile
\label{chapter:appendixC}
\markboth{Appendix C Additional SANS derivations}{Appendix C Additional SANS derivations}
\documentclass[../main.tex]{subfiles}
% ---------------------------------------------------------------- 
\begin{document}
\section{The spanwise-averaged vorticity transport equation}
\label{sec:savte}
The non-dimensional vorticity transport equation (VTE) can be written in its conservation form as
\begin{equation}
\partial_t\boldsymbol\omega+\nabla \cdot \pars{\vect{u}\otimes\boldsymbol{\omega}}=\nabla \cdot \pars{\boldsymbol{\omega}\otimes\vect{u}}+Re^{-1}\nabla^2\boldsymbol{\omega}.\label{eq:VTE}
\end{equation}
Decomposing and averaging similarly to \sref{sec:sans_math}, the following spanwise-averaged VTE can be obtained for the $\vect{e}_z$ component of \eref{eq:VTE},
\begin{gather}
\partial_t {\Omega}_z + U\partial_x{\Omega}_z + V\partial_y{\Omega}_z = Re^{-1}\pars{\partial_{xx}{\Omega}_z + \partial_{yy}{\Omega}_z}-\nonumber \\
-\partial_x\pars{\avg{u\p\omega_z\p}-\avg{\omega_x\p w\p}}-\partial_y\pars{\avg{v\p\omega_z\p}-\avg{\omega_y\p w\p}}+\nonumber \\
+{\Omega}_x\partial_xW+{\Omega}_y\partial_yW+{\Omega}_z\avg{\partial_z w\p}-W\avg{\partial_z\omega_z\p}+Re^{-1}\avg{\partial_{zz}\omega_z\p}, \label{eq:vort_e3_avg}
\end{gather}
where the spanwise-averaged divergence of the velocity and vorticity vector fields, respectively
\begin{gather}
\partial_x{U}+\partial_y{V}=-\avg{\partial_z{w\p}},\\
\partial_x{{\Omega}_x}+\partial_y{{\Omega}_y}=-\avg{\partial_z{\omega_z\p}},
\end{gather}
has been introduced to obtain the non-conservation form.

When spanwise periodicity is assumed, the third line in \eref{eq:vort_e3_avg} vanishes by its last three terms cancelling by definition (\eref{eq:periodic_assumption}), and the first two terms cancelling each other as follows,
\begin{gather}
{\Omega}_x\partial_xW+{\Omega}_y\partial_yW=\avg{\partial_y{w}-\partial_z{v}}\partial_x{W}+\avg{\partial_z{u}-\partial_z{w}}\partial_y{W}=\nonumber\\
=\avg{\partial_y{\pars{W+w\p}}-\partial_z{\pars{V+v\p}}}\partial_x{W}+\avg{\partial_z{\pars{U+u\p}}-\partial_z{\pars{W+v\p}}}\partial_y{W}=\nonumber\\
=\pars{\partial_y{W}-\cancelto{0}{\avg{\partial_z{v\p}}}}\partial_x{W}+\pars{\cancelto{0}{\avg{\partial_z{u\p}}}-\partial_x{W}}\partial_y{W}=0.
\end{gather}

The spanwise-averaged VTE is then simplified to
\begin{equation}
\partial_t {\Omega}_z + \vect{U}\cdot\nabla{\Omega}_z = Re^{-1}\nabla^2{\Omega}_z-\nabla\cdot\boldsymbol\zeta^R,
\end{equation}
where $\vect{U}=\pars{U,V}$ and $\boldsymbol \zeta^R = \pars{\avg{u\p\omega_z\p}-\avg{\omega_x\p w\p}, \avg{v\p\omega_z\p}-\avg{\omega_y\p w\p}}$.

\subsection{Cylindrical coordinates} \label{sec:cyl_coords}

In a cylindrical coordinates system, using $(\vect{e}_r,\vect{e}_\theta,\vect{e}_z)$ and
\begin{equation}
\nabla = \pars{\frac{1}{r}\pd{\pars{r\cdot}}{r}, \frac{1}{r}\pd{\pars{\cdot}}{\theta}, \pd{\pars{\cdot}}{z}},
\end{equation}
the $\vect{e}_z$ component of the non-dimensional VTE (\eref{eq:VTE}) can be written as
\begin{gather}
\pd{\omega_z}{t}+\frac{1}{r}\pd{\pars{r u_r\omega_z}}{r}+\frac{1}{r}\pd{\pars{u_\theta\omega_z}}{\theta}+\cancel{\pd{\pars{u_r\omega_z}}{z}}=\nonumber\\
=\frac{1}{r}\pd{\pars{r \omega_r u_z}}{r}+\frac{1}{r}\pd{\pars{\omega_\theta u_z}}{\theta}+\cancel{\pd{\pars{\omega_z u_r}}{z}}+\frac{1}{Re}\pars{\ddn{\omega_z}{r}{2}+\frac{1}{r}\pd{\omega_z}{r}+\frac{1}{r^2}\ddn{\omega_z}{\theta}{2}+\ddn{\omega_z}{z}{2}}.
\label{eq:VTE_cyl}
\end{gather}

A decomposition and average similarly to \sref{sec:sans_math} yields
\begin{gather}
\pd{{\Omega}_z}{t}+\frac{1}{r}\pd{\pars{r U_r{\Omega}_z+\avg{r u_r\p\omega_z\p}}}{r}+\frac{1}{r}\pd{\pars{U_\theta{\Omega}_z+\avg{u_\theta\p\omega_z\p}}}{\theta}=\nonumber\\
=\frac{1}{r}\pd{\pars{r {\Omega}_r U_z + \avg{r\omega_r\p u_z\p}}}{r}+\frac{1}{r}\pd{\pars{{\Omega}_\theta U_z+\avg{\omega_\theta\p u_z\p}}}{\theta}+\nonumber\\
+\frac{1}{Re}\pars{\ddn{{\Omega}_z}{r}{2}+\frac{1}{r}\pd{{\Omega}_z}{r}+\frac{1}{r^2}\ddn{{\Omega}_z}{\theta}{2}+\avg{\ddn{\omega_z\p}{z}{2}}}.
\label{eq:VTE_cyl_avg}
\end{gather}

The spanwise-averaged divergence of the velocity and vorticity fields can be respectively written in cylindrical coordinates as
\begin{gather}
\frac{1}{r}\pd{\pars{rU_r}}{r}+\frac{1}{r}\pd{U_\theta}{\theta}=-\avg{\pd{u_z\p}{z}},\label{eq:cont_avg_cyl}\\
\frac{1}{r}\pd{\pars{r{\Omega}_r}}{r}+\frac{1}{r}\pd{{\Omega}_\theta}{\theta}=-\avg{\pd{\omega_z\p}{z}}\label{eq:divort_avg_cyl},
\end{gather}
and introducing \eref{eq:cont_avg_cyl} and \eref{eq:divort_avg_cyl} into \eref{eq:VTE_cyl_avg} yields
\begin{gather}
\pd{{\Omega}_z}{t}+U_r\pd{{\Omega}_z}{r}+\frac{U_\theta}{r}\pd{{\Omega}_z}{\theta}={\Omega}_r\pd{U_z}{r}+\frac{{\Omega}_\theta}{r}\pd{U_z}{\theta}+\frac{1}{Re}\pars{\ddn{{\Omega}_z}{r}{2}+\frac{1}{r}\pd{{\Omega}_z}{r}+\frac{1}{r^2}\ddn{{\Omega}_z}{\theta}{2}}-\nonumber\\
-\frac{1}{r}\bracs{\pd{}{r}\pars{\avg{r u_r\p\omega_z\p}-\avg{r \omega_r\p u_z\p}}+
\pd{}{\theta}\pars{\avg{u_\theta\p\omega_z\p}-\avg{\omega_\theta\p u_z\p}}}+\nonumber\\
+{\Omega}_z\avg{\pd{u_z\p}{z}}-U_z\avg{\pd{\omega_z\p}{z}}+\frac{1}{Re}\avg{\ddn{\omega_z\p}{z}{2}}.
\end{gather}

\subsection{Relationship between SANS and spanwise-averaged VTE formulations}

Next, we show how the spanwise stresses in the SANS and the spanwise-averaged VTE frameworks are related.
First, we derive an expression for the vorticity-velocity terms of the spanwise-averaged VTE to write them as velocity-velocity terms.
Considering the following vector identity
\begin{equation}
\frac{1}{2} \nabla \left( \vect{a}\cdot\vect{a} \right) = (\vect{a} \cdot \nabla) \vect{a} + \vect{a} \times (\nabla \times \vect{a}),
\end{equation}
we can rewrite the vector identity in terms of fluctuating quantities
\begin{equation}
\frac{1}{2} \nabla \left( \vect{u}\p\cdot\vect{u}\p \right) = (\vect{u}\p \cdot \nabla) \vect{u}\p + \vect{u}\p \times \boldsymbol{\omega}\p,\label{eq:vect_identity_fluct}
\end{equation}
where the definition of the vorticity vector field, $\vect{\omega}=\nabla\times\vect{u}$, has been introduced.

Second, the conservation form of the convective forces can be expressed as
\begin{equation}
\nabla\cdot\pars{\vect u \otimes\vect u}=\pars{\vect u \cdot \nabla}\vect u + \vect u\pars{\nabla\cdot\vect u},
\end{equation}
and it can be rewritten for the fluctuating quantities as
\begin{gather}
\nabla\cdot\pars{\vect u\p \otimes\vect u\p}=\pars{\vect u\p \cdot \nabla}\vect u\p + \vect u\p\pars{\nabla\cdot\vect u\p},\\
\pars{\vect u\p \cdot \nabla}\vect u\p = \nabla\cdot\pars{\vect u\p \otimes\vect u\p}-\vect u\p\pars{\nabla\cdot\vect u\p}.\label{eq:conv_term_fluct}
\end{gather}

Third, decomposing the continuity equation for incompressible flows (\eref{eq:cont})
\begin{equation}
\partial_xU+\partial_xu\p+\partial_yV+\partial_yv\p+\partial_zw\p=0,\label{eq:continuity_decomp}
\end{equation}
and subtracting the averaged continuity equation (\eref{eq:cont_decomp_avg}) yields
\begin{equation}
\partial_xu\p+\partial_yv\p+\partial_zw\p-\avg{\partial_z w\p}=0,
\end{equation}
hence
\begin{equation}
\nabla\cdot\vect u\p =\avg{\partial_z w\p}.\label{eq:cont_fluct}
\end{equation}

We now introduce \eref{eq:cont_fluct} into \eref{eq:conv_term_fluct} obtaining
\begin{equation}
\pars{\vect u\p \cdot \nabla}\vect u\p = \nabla\cdot\pars{\vect u\p \otimes\vect u\p}-\vect u\p\avg{\partial_z w\p},\label{eq:conv_term_fluct2}
\end{equation}
and \eref{eq:conv_term_fluct2} is finally introduced into \eref{eq:vect_identity_fluct}
\begin{equation}
\frac{1}{2} \nabla \left( \vect{u}\p\cdot\vect{u}\p \right) = \nabla\cdot\pars{\vect u\p \otimes\vect u\p}-\vect u\p\avg{\partial_z w\p} + \vect{u}\p \times \boldsymbol{\omega}\p,
\end{equation}
which yields
\begin{equation}
\boldsymbol{\omega}\p\times\vect{u}\p = \nabla\cdot\pars{\vect u\p \otimes\vect u\p} - \frac{1}{2} \nabla \left( \vect{u}\p\cdot\vect{u}\p \right) - \vect u\p\avg{\partial_z w\p}.
\label{eq:vect_identity_fluct2}
\end{equation}

With this, a relation between the vorticity-velocity spanwise stresses and the velocity-velocity spanwise stresses is defined in the conservation form.
To derive a more explicit form of the spanwise-averaged stresses, let us average \eref{eq:vect_identity_fluct2} and retrieve its $\vect{e}_x$ and $\vect{e}_y$ components
\begin{gather}
\vect{e}_x: \quad \avg{\omega_y\p w\p}-\avg{\omega_z\p v\p} = \partial_x\avg{u\p u\p}+\partial_y\avg{v\p u\p}+\avg{\partial_z\pars{w\p u\p}}-\nonumber\\
-\partial_x\pars{\avg{u\p u\p}+\avg{v\p v\p}+\avg{w\p w\p}}/2,\label{eq:vec_ident_fluct_e1}\\
\vect{e}_y: \quad \avg{\omega_z\p u\p}-\avg{\omega_x\p w\p} = \partial_x\avg{u\p v\p}+\partial_y\avg{v\p v\p}+\avg{\partial_z\pars{w\p v\p}}-\nonumber\\
-\partial_y\pars{\avg{u\p u\p}+\avg{v\p v\p}+\avg{w\p w\p}}/2\label{eq:vec_ident_fluct_e2}.
\end{gather}
Note that the last term of \eref{eq:vect_identity_fluct2} vanishes because of \eref{eq:rule2}. 

The spanwise stresses of the spanwise-averaged VTE (\eref{eq:vort_e3_avg}) can be expanded using Eqs. \ref{eq:vec_ident_fluct_e1} and \ref{eq:vec_ident_fluct_e2} as follows
\begin{gather}
-\partial_x\pars{\avg{u\p\omega_z\p}-\avg{\omega_x\p w\p}}-\partial_y\pars{\avg{v\p\omega_z\p}-\avg{\omega_y\p w\p}} = \nonumber\\
=-\partial_x\pars{\partial_x\avg{u\p v\p}+\partial_y\avg{v\p v\p}+\avg{\partial_z\pars{w\p v\p}} - \partial_y\pars{\avg{u\p u\p}+\avg{v\p v\p}+\avg{w\p w\p}}/2} +\nonumber\\
+\partial_y\pars{\partial_x\avg{u\p u\p}+\partial_y\avg{v\p u\p}+\avg{\partial_z\pars{w\p u\p}} - \partial_x\pars{\avg{u\p u\p}+\avg{v\p v\p}+\avg{w\p w\p}}/2}=\nonumber\\
=-\partial_{xx}\avg{u\p v\p}-\partial_{xy}\avg{v\p v\p}-\partial_x\avg{\partial_z\pars{w\p v\p}}+\partial_{xy}\pars{\avg{u\p u\p}+\avg{v\p v\p}+\avg{w\p w\p}}/2+\nonumber\\
+\partial_{xy}\avg{u\p u\p}+\partial_{yy}\avg{v\p u\p}+\partial_y\avg{\partial_z\pars{w\p u\p}}-\partial_{xy}\pars{\avg{u\p u\p}+\avg{v\p v\p}+\avg{w\p w\p}}/2=\nonumber\\
=\pars{\partial_{yy}-\partial_{xx}}\avg{u\p v\p}+\partial_{xy}\pars{\avg{u\p u\p}-\avg{v\p v\p}}+\partial_y\avg{\partial_z\pars{w\p u\p}}-\partial_x\avg{\partial_z\pars{w\p v\p}}\label{eq:vte_spanwise_stresses}.
\end{gather}

On the other hand, the SANS equations (Eqs. \ref{eq:z-avg_X} to \ref{eq:z-avg_Z}) have yield the following spanwise stresses
\begin{equation}
-\avg{\nabla\cdot\pars{\vect u\p\otimes\vect u\p}}=
-\begin{pmatrix}
    \partial_x\avg{u\p u\p}+\partial_y\avg{u\p v\p}+\avg{\partial_z\pars{w\p u\p}} \\
    \partial_x\avg{v\p u\p}+\partial_y\avg{v\p v\p}+\avg{\partial_z\pars{w\p v\p}} \\
    \partial_x\avg{w\p u\p}+\partial_y\avg{w\p v\p}+\avg{\partial_z\pars{w\p w\p}}
  \end{pmatrix}.
\end{equation}
Taking the curl of the momentum spanwise stresses
\begin{equation}
-\nabla\times\avg{\nabla\cdot\pars{\vect u\p\otimes\vect u\p}}=-
\begin{pmatrix}
    \vect{e}_x & \partial_x & \partial_x\avg{u\p u\p}+\partial_y\avg{u\p v\p}+\avg{\partial_z\pars{w\p u\p}} \\
    \vect{e}_y & \partial_y & \partial_x\avg{v\p u\p}+\partial_y\avg{v\p v\p}+\avg{\partial_z\pars{w\p v\p}} \\
    \vect{e}_z & \partial_z & \partial_x\avg{w\p u\p}+\partial_y\avg{w\p v\p}+\avg{\partial_z\pars{w\p w\p}}
  \end{pmatrix},
\end{equation}
recovers the VTE spanwise stresses as demonstrated for the $\vect{e}_z$ component
\begin{gather}
-\nabla\times\avg{\nabla\cdot\pars{\vect u\p\otimes\vect u\p}}\vect{e}_z=\nonumber\\
=-\partial_x\pars{\partial_x\avg{v\p u\p}+\partial_y\avg{v\p v\p}+\avg{\partial_z\pars{w\p v\p}}}
+\partial_y\pars{\partial_x\avg{u\p u\p}+\partial_y\avg{u\p v\p}+\avg{\partial_z\pars{w\p u\p}}}=\nonumber\\
=-\partial_{xx}\avg{v\p u\p}-\partial_{xy}\avg{v\p v\p}-\partial_x\avg{\partial_z\pars{w\p v\p}}+\partial_{xy}\avg{u\p u\p}+\partial_{yy}\avg{u\p v\p}+\partial_y\avg{\partial_z\pars{w\p u\p}}=\nonumber\\
=\pars{\partial_{yy}-\partial_{xx}}\avg{u\p v\p}+\partial_{xy}\pars{\avg{u\p u\p}-\avg{v\p v\p}}+\partial_y\avg{\partial_z\pars{w\p u\p}}-\partial_x\avg{\partial_z\pars{w\p v\p}},\label{eq:NS_spanwise_stresses}
\end{gather}
showing that the spanwise stresses in both formulations (Eqs. \ref{eq:vte_spanwise_stresses} and \ref{eq:NS_spanwise_stresses}) are equivalent.

\section{Analytical validation and quadrature error analysis}

The aim of this exercise is to analytically demonstrate the correct derivation of the SANS equations.
The velocity vector field needs to be solenoidal and periodic, in agreement with the assumption used for the derivation of the simplified SANS equations (\eref{eq:sans_simple}). For this, the Taylor--Green vortex case is considered in a $2\pi$-periodic box.
The pressure field is taken constant.
The components of its periodic solenoidal initial velocity vector field might be written as
\begin{gather}
u = \cos (x) \sin (y) \sin (z), \label{eq:tg-u} \\
v = \sin (x) \cos (y) \sin (z), \\
w = -2 \sin (x) \sin (y) \cos (z),\label{eq:tg-w}
\end{gather}
which is indeed a solenoidal vector field
\begin{gather}
\nabla \cdot \vect{u} = \pd{u}{x}+\pd{v}{y}+\pd{w}{z}=\nonumber\\
=- \sin (x) \sin (y) \sin (z)- \sin (x) \sin (y) \sin (z)+2\sin (x) \sin (y) \sin (z)=0.
\end{gather}

\subsection{No quadrature errors and spanwise periodicity}
\label{sec:sans_analytical}

The averaging operation in the spanwise direction $\avg{\cdot}$ is defined on the $z\in\left[a,b\right]$ interval as
\begin{equation}
\avg{q}\pars{x,y,t} = \frac{1}{b-a}\int^b_a q\pars{x,y,z,t}\,dz.
\end{equation}
Numerically, this carries a quadrature error which we will neglect for the moment.
The following condition needs to hold true assuming an error-free quadrature and spanwise periodicity,
\begin{equation}
\avg{\mathcal{S}\left(\vect{u}, p\right)} = \tilde{\mathcal{S}}\left(\vect{U}, P\right) + \nabla\cdot\boldsymbol\tau_{ij}^R, \label{eq:balance}
\end{equation}
where
\begin{align}
\mathcal{S}= \vect{u}\cdot\nabla \vect{u} + \nabla p - \nu \nabla^2 \vect{u}, \qquad &\vect{u}=\pars{u,v,w},\,\,\, \vect{x}=\pars{x,y,z},\\
 \tilde{\mathcal{S}}= \vect{U}\cdot\nabla \vect{U} + \nabla P - \nu \nabla^2 \vect{U}, \qquad &\vect{U} = \pars{U,V},\,\,\, \vect{x}=\pars{x,y}.
\end{align}

The derivation to validate the SANS equations will be performed for the $\vect{e}_x$ component of \eref{eq:balance}.
In a $L=b-a=2\pi$ domain and considering the velocity vector field defined in \eref{eq:tg-u} to \eref{eq:tg-w}, we obtain $\vect{U}=\pars{0,0,0}$, so $\vect{u}=\vect{u}\p$.

We begin calculating the LHS of \eref{eq:balance},
\begin{equation}
\avg{\mathcal{S}}\vect{e}_x=\avg{u\pd{u}{x}+v\pd{u}{y}+w\pd{u}{z}+\pd{p}{x}-\nu\pars{\ddn{u}{x}{2}+\ddn{u}{y}{2}+\ddn{u}{z}{2}}}.\label{eq:LHS}
\end{equation}
The required derivatives are
\begin{gather}
\begin{align}
\pd{u}{x}&=-\sin (x) \sin (y) \sin (z), \qquad \ddn{u}{x}{2}=-\cos (x) \sin (y) \sin (z)=-u,\\
\pd{u}{y}&= \cos (x) \cos (y) \sin (z), \qquad\,\,\,\,\, \ddn{u}{y}{2}=-\cos (x) \sin (y) \sin (z)=-u,\\
\pd{u}{z}&=\cos (x) \sin (y) \cos (z), \qquad\,\,\,\,\, \ddn{u}{z}{2}=-\cos (x) \sin (y) \sin (z)=-u.
\end{align}
\end{gather}

Substituting the derivatives into \eref{eq:LHS} yields
\begin{gather}
\avg{\mathcal{S}}\vect{e}_x=\avg{u\pd{u}{x}+v\pd{u}{y}+w\pd{u}{z}+\pd{p}{x}-\nu\pars{\ddn{u}{x}{2}+\ddn{u}{y}{2}+\ddn{u}{z}{2}}}= \nonumber \\
=\bigl\langle-\sin (x) \cos (x) \sin^2 (y) \sin^2 (z) + \sin (x) \cos (x) \cos^2 (y) \sin^2 (z)-\nonumber\\
\quad\;\;-2\sin (x) \cos (x) \sin^2 (y) \cos^2 (z) + 3\nu u \bigr\rangle = \nonumber\\
=-\frac{\pi}{2\pi}\sin (x) \cos (x) \sin^2 (y) + \frac{\pi}{2\pi} \sin (x) \cos (x) \cos^2 (y) -\frac{2\pi}{2\pi} \sin (x) \cos (x) \sin^2 (y) + 0=\nonumber\\
=\frac{1}{2}\sin(x) \cos(x) \pars{\cos^2(y)-3\sin^2(y)}.
\label{eq:b21_lhs}
\end{gather}

Next, we calculate the RHS of \eref{eq:balance}. Note that $\tilde{\mathcal{S}}=0$ because $\vect{U}=0$ and the pressure is constant.
Hence, only the spanwise stresses remain, analytically
\begin{gather}
\pars{\nabla\cdot\boldsymbol\tau_{ij}^R}\vect{e}_x = \pd{}{x}\avg{u\p u\p}+\pd{}{y}\avg{u\p v\p} = \pd{}{x}\avg{u u}+\pd{}{y}\avg{u v} = \nonumber \\
=\pd{}{x}\pars{\avg{\cos^2(x) \sin^2(y) \sin^2(z)}}+\pd{}{y}\pars{\avg{\sin(x)\cos(x)\sin(y)\cos(y)\sin^2(z)}}=\nonumber \\
=\pd{}{x}\pars{ \frac{\pi }{2\pi }\cos^2(x) \sin^2(y)}+\pd{}{y}\pars{\frac{\pi }{2\pi }\sin(x)\cos(x)\sin(y)\cos(y)}=\nonumber \\
=-\sin(x)\cos(x)\sin^2(y) +\frac{1}{2}\sin(x)\cos(x)\pars{\cos^2(x)-\sin^2(x)}=\nonumber \\
=\frac{1}{2} \sin(x) \cos(x) \pars{\cos^2(y)-3\sin^2(y)}.
\label{eq:b21_rhs}
\end{gather}

Finally, we can show that \eref{eq:balance} has the correct balance between both sides by bringing all the pieces together, i.e. using \eref{eq:b21_lhs} as LHS and \eref{eq:b21_rhs} as RHS,
\begin{equation}
\frac{1}{2} \sin(x) \cos(x) \pars{\cos^2(y)-3\sin^2(y)} = 0 + \frac{1}{2} \sin(x) \cos(x) \pars{\cos^2(y)-3\sin^2(y)}.
\end{equation}

\subsection{Quadrature errors without spanwise periodicity}
\label{sec:sans_quadrature_errors1}

When quadrature errors arising from the averaging operation are considered, we have to recall the second-line terms in \eref{eq:sans_full} that were neglected in the previous section.
The quadrature error depends on the function being averaged and the quadrature itself.
From here, we will consider the composite trapezoidal quadrature which has the following associated error on the integral interval $[a,b]$
\begin{equation}
\epsilon_T(\xi) = -\frac{(b-a)h^2}{12} f''(\xi), \qquad \xi \in [a,b],
\end{equation}
where $h$ is the subinterval (spacing) length defined as $h=(b-a)/N$, and $N$ is the number of subintervals for $N+1$ quadrature points.
Note that the convergence error is $\mathcal{O}\pars{h^2}$ and the presence of $f''$, which implies that the trapezoidal rule can perfectly calculate the quadrature of zero and first degree polynomial functions.

For the Taylor--Green vortex case, only first- and second-degree trigonometric functions are required. We can quantify the error associated to its quadrature using the definition above in the $2\pi$ interval as
\begin{gather}
I_1 =\int^{2\pi}_0 \sin(z) dz, \\
\epsilon_1\pars{\xi} := I_1 - I_{1,N} = \frac{\pi h^2}{6} \sin(\xi),
\end{gather}
and
\begin{gather}
I_2 =\int^{2\pi}_0 \sin^2(z) dz, \\
\epsilon_2\pars{\xi} := I_2 - I_{2,N} = -\frac{\pi h^2}{6} \bracs{2\pars{\cos^2(\xi)-\sin^2(\xi)}}.
\end{gather} 
Next, we bound $\epsilon_1$ and $\epsilon_2$ with $|f''(\xi)|_{\max}$ within $[0,2\pi]$,
\begin{gather}
|\epsilon_1|_{\max} = \frac{\pi h^2}{6} \cdot 1, \\
|\epsilon_2|_{\max} = \frac{\pi h^2}{6} \cdot 2,
\end{gather}
hence
\begin{equation}
|\epsilon_2|_{\max}= 2|\epsilon_1|_{\max}.
\end{equation}
Defining $I_1^*$ and $I_2^*$ as $I^* = I/(2\pi)$, we obtain
\begin{gather}
I^*_1 = \frac{0+\epsilon}{2\pi}=\frac{\epsilon}{2\pi}, \\
I^*_2 = \frac{\pi+2\epsilon}{2\pi}=\frac{1}{2}+\frac{\epsilon}{\pi},
\end{gather}
where $\epsilon = \pi h^2/6$. Note that $I^*_1$ and $I^*_2$ using $\cos(z)$ instead of $\sin(z)$ are similarly calculated. 

Considering quadrature errors, \eref{eq:balance} has to be corrected with the terms previously neglected because of spanwise periodicity,
\begin{equation}
\avg{\mathcal{S}\left(\vect{u}, p\right)} = \tilde{\mathcal{S}}\left(\vect{U}, P\right) + \nabla\cdot\boldsymbol\tau_{ij}^R+W\avg{\partial_z\vect{u}\p}-\nu\avg{\partial_{zz}\vect{u}\p}+\avg{\partial_z\pars{w\p\vect{u}\p}}.\label{eq:balance_all}
\end{equation}

We proceed by calculating again the new average and fluctuating quantities using the errors described above as follows
\begin{gather}
U = \frac{\epsilon}{2\pi}\cos(x)\sin(y),\qquad u\p=\cos(x)\sin(y)\pars{\sin(z)-\frac{\epsilon}{2\pi}},\\
V = \frac{\epsilon}{2\pi}\sin(x)\cos(y),\qquad v\p=\sin(x)\cos(y)\pars{\sin(z)-\frac{\epsilon}{2\pi}},\\
W = -\frac{\epsilon}{\pi}\sin(x)\sin(y),\,\,\,\,\, w\p=2\sin(x)\sin(y)\pars{\frac{\epsilon}{2\pi}-\cos(z)}.
\end{gather}
With this, the LHS of \eref{eq:balance_all} can be written as
\begin{gather}
\avg{\mathcal{S}}\vect{e}_x=\avg{u\pd{u}{x}+v\pd{u}{y}+w\pd{u}{z}+\pd{p}{x}-\nu\pars{\ddn{u}{x}{2}+\ddn{u}{y}{2}+\ddn{u}{z}{2}}}=\nonumber \\ 
=\bigl\langle-\sin (x) \cos (x) \sin^2 (y) \sin^2 (z) + \sin (x) \cos (x) \cos^2 (y) \sin^2 (z)-\nonumber\\
\quad\;\;-2\sin (x) \cos (x) \sin^2 (y) \cos^2 (z) + 3\nu u \bigr\rangle = \nonumber\\
=-\pars{\frac{1}{2}+\frac{\epsilon}{\pi}}\sin(x)\cos(x)\sin^2(y) +\pars{\frac{1}{2}+\frac{\epsilon}{\pi}}\sin(x)\cos(x)\cos^2(y) \nonumber \\
-2\pars{\frac{1}{2}+\frac{\epsilon}{\pi}}\sin(x)\cos(x)\sin^2(y)+3\nu\frac{\epsilon}{2\pi}\cos(x)\sin(y) = \nonumber\\
=\pars{\frac{1}{2}+\frac{\epsilon}{\pi}}\sin(x)\cos(x)\pars{\cos^2(y)-3\sin^2(y)}+3\nu\frac{\epsilon}{2\pi}\cos(x)\sin(y).\label{eq:S_avg}
\end{gather}

The spatial operator of the spanwise-averaged quantities $(\tilde{\mathcal{S}})$ considering quadrature errors can be written as
\begin{gather}
\tilde{\mathcal{S}}\,\vect{e}_x=U\pd{U}{x}+V\pd{U}{y}+\pd{P}{x}-\nu\pars{\ddn{U}{x}{2}+\ddn{U}{y}{2}}=\nonumber\\
=\frac{\epsilon}{2\pi}\cos(x)\sin(y)\pd{}{x}\pars{\frac{\epsilon}{2\pi}\cos(x)\sin(y)}+\frac{\epsilon}{2\pi}\sin(x)\cos(y)\pd{}{y}\pars{\frac{\epsilon}{2\pi}\cos(x)\sin(y)}+0-\nonumber\\
-\nu\bracs{\ddn{}{x}{2}\pars{\frac{\epsilon}{2\pi}\cos(x)\sin(y)}+\ddn{}{y}{2}\pars{\frac{\epsilon}{2\pi}\cos(x)\sin(y)}}=\nonumber\\
=-\pars{\frac{\epsilon}{2\pi}}^2\sin(x)\cos(x)\sin^2(x)+\pars{\frac{\epsilon}{2\pi}}^2\sin(x)\cos(x)\cos^2(y)+2\nu\frac{\epsilon}{2\pi}\cos(x)\sin(y)=\nonumber\\
=\pars{\frac{\epsilon}{2\pi}}^2\sin(x)\cos(x)\pars{\cos^2(y)-\sin^2(x)}+\nu\frac{\epsilon}{\pi}\cos(x)\sin(y).\label{eq:S_tilde}
\end{gather}

And $\pars{\nabla\cdot\boldsymbol\tau_{ij}^R}\vect{e}_x$ is calculated as follows
\begin{gather}
\pars{\nabla\cdot\boldsymbol\tau_{ij}^R}\vect{e}_x=\pd{}{x}\avg{u\p u\p}+\pd{}{y}\avg{u\p v\p} =\nonumber\\
=\pd{}{x}\pars{\avg{\cos^2(x) \sin^2(y) \pars{\sin(z)-\frac{\epsilon}{2\pi}}^2}}+\nonumber\\
+\pd{}{y}\pars{\avg{\sin(x)\cos(x)\sin(y)\cos(y)\pars{\sin(z)-\frac{\epsilon}{2\pi}}^2}}=\nonumber \\
=\pd{}{x}\pars{\cos^2(x) \sin^2(y)\avg{ \pars{\sin^2(z)+\pars{\frac{\epsilon}{2\pi}}^2-\frac{\epsilon}{\pi}\sin(z)}}}+\nonumber\\
+\pd{}{y}\pars{\sin(x)\cos(x)\sin(y)\cos(y)\avg{\pars{\sin^2(z)+\pars{\frac{\epsilon}{2\pi}}^2-\frac{\epsilon}{\pi}\sin(z)}}}=\nonumber \\
=\bracs{\frac{1}{2}+\frac{\epsilon}{\pi}+\pars{\frac{\epsilon}{2\pi}}^2-\frac{1}{2}\pars{\frac{\epsilon}{\pi}}^2}\Bigg[\pd{}{x}\pars{\cos^2(x) \sin^2(y)}+\nonumber\\
+\pd{}{y}\pars{\sin(x)\cos(x)\sin(y)\cos(y)}\Bigg]=\nonumber\\
=\bracs{\frac{1}{2}+\frac{\epsilon}{\pi}-\frac{1}{4}\pars{\frac{\epsilon}{\pi}}^2}\sin(x)\cos(x)\pars{\cos^2(y)-3\sin^2(y)}.\label{eq:divT}
\end{gather}

Next, we calculate the terms neglected in the previous section
\begin{gather}
W\avg{\partial_z\vect{u}\p}-\nu\avg{\partial_{zz}\vect{u}\p}+\avg{\partial_z\pars{w\p\vect{u}\p}}=\nonumber \\
=-\frac{\epsilon}{\pi}\sin(x)\sin(y)\avg{\pd{}{z}\pars{\cos(x)\sin(y)\pars{\sin(z)-\frac{\epsilon}{2\pi}}}}-\nonumber \\ 
-\nu\avg{\ddn{}{z}{2}\pars{\cos(x)\sin(y)\pars{\sin(z)-\frac{\epsilon}{2\pi}}}}+\nonumber \\
+\avg{\pd{}{z}\pars{\cos(x)\sin(y)\pars{\sin(z)-\frac{\epsilon}{2\pi}}2\sin(x)\sin(y)\pars{\frac{\epsilon}{2\pi}-\cos(z)}}}=\nonumber
\end{gather}
\begin{gather}
=-\frac{\epsilon}{\pi}\sin(x)\cos(x)\sin^2(y)\avg{\cos(z)}+\nu\cos(x)\sin(y)\avg{\sin(z)}+\nonumber\\
+2\sin(x)\cos(x)\sin^2(y)\avg{\pd{}{z}\pars{\frac{\epsilon}{2\pi}\sin(z)+\frac{\epsilon}{2\pi}\cos(z)-\sin(z)\cos(z)-\pars{\frac{\epsilon}{2\pi}}^2}}=\nonumber\\
=-\frac{1}{2}\pars{\frac{\epsilon}{\pi}}^2\sin(x)\cos(x)\sin^2(y)+\nu\frac{\epsilon}{2\pi}\cos(x)\sin(y)+\nonumber\\
+2\sin(x)\cos(x)\sin^2(y)\avg{\frac{\epsilon}{2\pi}\cos(z)-\frac{\epsilon}{2\pi}\sin(z)+\sin^2(z)-\cos^2(z)}=\nonumber\\
=-\frac{1}{2}\pars{\frac{\epsilon}{\pi}}^2\sin(x)\cos(x)\sin^2(y)+\nu\frac{\epsilon}{2\pi}\cos(x)\sin(y)+\nonumber\\
+2\sin(x)\cos(x)\sin^2(y)\cancelto{0}{\bracs{\pars{\frac{\epsilon}{2\pi}}^2-\pars{\frac{\epsilon}{2\pi}}^2+\frac{1}{2}+\frac{\epsilon}{\pi}-\frac{1}{2}-\frac{\epsilon}{\pi}}}=\nonumber\\
=-\frac{1}{2}\pars{\frac{\epsilon}{\pi}}^2\sin(x)\cos(x)\sin^2(y)+\nu\frac{\epsilon}{2\pi}\cos(x)\sin(y).\label{eq:extra}
\end{gather}
We note that $\avg{\partial_z\pars{w\p\vect{u}\p}}$ automatically cancels out.

Putting all the pieces together (Eqs. \ref{eq:S_avg}, \ref{eq:S_tilde}, \ref{eq:divT} \& \ref{eq:extra} into \eref{eq:balance_all}) we obtain the error between the balance of its LHS and RHS as follows
\begin{gather}
\mathcal{E} = \left|\avg{\mathcal{S}\left(\vect{u}, p\right)} - \tilde{\mathcal{S}}\left(\vect{U}, P\right) - \nabla\cdot\boldsymbol\tau_{ij}^R - W\avg{\partial_z\vect{u}\p}+\nu\avg{\partial_{zz}\vect{u}\p} - \avg{\partial_z\pars{w\p\vect{u}\p}}\right|=\nonumber\\
=\Bigg|\cancel{\pars{\frac{1}{2}+\frac{\epsilon}{\pi}}\sin(x)\cos(x)\pars{\cos^2(y)-3\sin^2(y)}}+\bcancel{3\nu\frac{\epsilon}{2\pi}\cos(x)\sin(y)} - \nonumber\\
- \pars{\frac{\epsilon}{2\pi}}^2\sin(x)\cos(x)\pars{\cos^2(y)-\sin^2(x)}-\bcancel{\nu\frac{\epsilon}{\pi}\cos(x)\sin(y)} - \nonumber \\
- \bracs{\cancel{\frac{1}{2}+\frac{\epsilon}{\pi}}-\frac{1}{4}\pars{\frac{\epsilon}{\pi}}^2}\sin(x)\cos(x)\pars{\cos^2(y)-3\sin^2(y)}+ \nonumber \\
+\frac{1}{2}\pars{\frac{\epsilon}{\pi}}^2\sin(x)\cos(x)\sin^2(y)-\bcancel{\nu\frac{\epsilon}{2\pi}\cos(x)\sin(y)}\Bigg| \nonumber=\\
=\Bigg|-\frac{1}{4}\pars{\frac{\epsilon}{\pi}}^2\sin(x)\cos(x)\pars{\cos^2(y)-\sin^2(x)}+\nonumber\\
+\frac{1}{4}\pars{\frac{\epsilon}{\pi}}^2\sin(x)\cos(x)\pars{\cos^2(y)-3\sin^2(y)}+\frac{1}{2}\pars{\frac{\epsilon}{\pi}}^2\sin(x)\cos(x)\sin^2(y)\Bigg| \nonumber= \\
=\Bigg|-\frac{1}{2}\pars{\frac{\epsilon}{\pi}}^2\sin(x)\cos(x)\sin^2(x)+\frac{1}{2}\pars{\frac{\epsilon}{\pi}}^2\sin(x)\cos(x)\sin^2(y)\Bigg|= 0.
\end{gather}

With this, we can conclude that the quadratic errors arising from the RHS of \eref{eq:balance_all} cancel out with each other.
Also, the linear errors arising on both LHS and RHS also cancel each other.

\subsection{Quadrature error with spanwise periodicity}
\label{sec:sans_quadrature_errors2}

The SANS derivation employed in \cref{chapter:sans} and \cref{chapter:zanspy} (\eref{eq:balance}) neglects errors arising from the quadratures of spanwise derivatives (second line of \eref{eq:sans_full}).
Because of this, \eref{eq:balance_all} is not balanced.
Here, we derive the error associated to this practice.
Inserting the previously derived terms into \eref{eq:balance} carries the following error
\begin{gather}
\mathcal{E} = \avg{\mathcal{S}\left(\vect{u}, p\right)} - \tilde{\mathcal{S}}\left(\vect{U}, P\right) - \nabla\cdot\boldsymbol\tau_{ij}^R=\nonumber\\
=\cancel{\pars{\frac{1}{2}+\frac{\epsilon}{\pi}}\sin(x)\cos(x)\pars{\cos^2(y)-3\sin^2(y)}}+3\nu\frac{\epsilon}{2\pi}\cos(x)\sin(y) - \nonumber\\
- \pars{\frac{\epsilon}{2\pi}}^2\sin(x)\cos(x)\pars{\cos^2(y)-\sin^2(x)}-{\nu\frac{\epsilon}{\pi}\cos(x)\sin(y)} - \nonumber \\
- \bracs{\cancel{\frac{1}{2}+\frac{\epsilon}{\pi}}-\frac{1}{4}\pars{\frac{\epsilon}{\pi}}^2}\sin(x)\cos(x)\pars{\cos^2(y)-3\sin^2(y)} \nonumber=\\
=\frac{\epsilon}{\pi}\nu\cos(x)\sin(y) - \frac{1}{4}\pars{\frac{\epsilon}{\pi}}^2\sin(x)\cos(x)\pars{\cos^2(y)-\sin^2(x)} + \nonumber \\
+\frac{1}{4}\pars{\frac{\epsilon}{\pi}}^2\sin(x)\cos(x)\pars{\cos^2(y)-3\sin^2(y)} \nonumber=\\
=\frac{\epsilon}{\pi}\nu\cos(x)\sin(y)-\frac{1}{2}\pars{\frac{\epsilon}{\pi}}^2\sin(x)\cos(x)\sin^2(y),
\end{gather}
which corresponds indeed to the derivation of \eref{eq:extra}
\begin{equation}
\mathcal{E}=|\avg{w}\avg{\partial_z\vect{u}\p}-\nu\avg{\partial_{zz}\vect{u}\p}|.
\end{equation}
It can be observed that, given $\epsilon$ is a small positive number, the viscous term has a larger contribution to the model error.

It is important to note that the midpoint and trapezoidal quadratures obtain a higher convergence rate when the integrand is periodic and smooth \citep{Weideman2002,Trefethen2014}, precisely from $\mathcal{O}(h^2)$ to $\mathcal{O}(e^{-1/h})$.
Applying the spanwise-periodicity constraint in our problem and considering incompressible flow (smooth functions), the quadrature error should converge faster than the spatial discretisation error of the governing equations $\pars{\mathcal{O}(h^2)}$.
% ---------------------------------------------------------------- 
\end{document}